\section*{Discussion}
% The Discussion section should again have 3 sub-sections for Q3, Q4 and Q5. In each section you should discuss the benefits (and drawbacks if any) of the approaches/architectures you used and some discussion of why you think you got the results you got.
% Please discuss the following important points as well: How did the performance of your implementations differ? Discuss the performance differences with respect to the baseline (part 3), improved baseline (part 4) and compare the other implementations (part 5a, 5b, 5c) and mention them in their respective sub-sections. Also, draw insights from the plotted loss curves, tables and the visualizations.

\subsection*{Baseline}
The baseline model, though straightforward to implement, exhibited limitations in segmentation performance with an IOU of 0.06 and a pixel accuracy of 75%. The simplicity of the architecture hindered its ability to capture intricate features in the data.

\subsection*{Improved Baseline (FCN with Augmentation)}
In an effort to address overfitting, the baseline model was enhanced with data augmentation. Despite these improvements, the model faced challenges with slower loss convergence and only achieved a marginal increase in performance, resulting in an IOU of 0.06 and a decreased pixel accuracy of 72%.

\subsection*{DarrenNet:}
DarrenNet, incorporating advanced architectural features and augmentation techniques such as Affine transformations, displayed notable improvements with an IOU of 0.15 and a pixel accuracy of 82%. The architecture and augmentation strategy played pivotal roles in enhancing segmentation results.

\subsection*{Transfer Darrennet:}
The application of transfer learning with DarrenNet, utilizing pre-trained weights from resnet34, demonstrated significant performance gains with an IOU of 0.10 and a pixel accuracy of 78%. Successful knowledge transfer from the source domain contributed to the improved segmentation performance.

\subsection*{UNet:}
The UNet architecture, characterized by its ability to capture detailed spatial information, showed below-average baseline performance with an IOU of 0.05 and a pixel accuracy of 55%. However, when integrated with transfer learning, UNet outperformed the baseline, achieving an IOU of 0.08 and a pixel accuracy of 79%.

\subsection*{Performance Differences (Between Implementations):}
Comparative analysis revealed that DarrenNet consistently outperformed the baseline, showcasing the significance of architectural enhancements. Transfer learning, as evidenced in both Transfer DarrenNet and UNet with transfer, played a critical role in improving segmentation results, emphasizing the importance of leveraging pre-trained weights for feature extraction.

\subsection*{Insights from Loss Curves, Tables, Visualizations:}
Examination of loss convergence in the improved baseline highlighted challenges in augmentation effectiveness. Visualizations, loss curves, and performance metrics provided insights into the impact of architectural choices and transfer learning strategies, offering valuable information for further model refinement.


% Baseline:
% - Benefits
% - Drawbacks
% - Why we got the results we got
%
% Improved baseline:
% - Benefits
% - Drawbacks
% - Why we got the results we got
%
% DarrenNet:
% - Benefits
% - Drawbacks
% - Why we got the results we got
%
% Transfer Darrennet:
% - Benefits
% - Drawbacks
% - Why we got the results we got
%
% UNet:
% - Benefits
% - Drawbacks
% - Why we got the results we got
%
% Performance Differences (between implementations)
%
% Insights from loss curves, tables, visualizations
